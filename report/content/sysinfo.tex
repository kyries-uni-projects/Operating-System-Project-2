\subsubsection{Mô tả bài toán}

Thêm một system call mới \texttt{sysinfo} để thu thập thông tin về hệ thống đang chạy.
Lệnh gọi hệ thống này nhận một đối số là con trỏ tới \texttt{struct sysinfo} 
(được định nghĩa trong \texttt{kernel/sysinfo.h}). Kernel sẽ điền vào các trường của struct này:
trường \texttt{freemem} sẽ được đặt thành số byte bộ nhớ trống và trường \texttt{nproc} 
sẽ được đặt thành số tiến trình có trạng thái không phải \texttt{UNUSED}.

Ngoài ra, viết một chương trình thử nghiệm \texttt{sysinfotest} để kiểm tra
tính đúng đắn của system call. Chương trình sẽ vượt qua bài tập nếu nó in ra
\texttt{"sysinfotest: OK"}.

\subsubsection{Phương pháp thực hiện}
Bài làm gồm các bước: (1) định nghĩa cấu trúc \texttt{sysinfo},
(2) cung cấp syscall \texttt{sysinfo} để thu thập và trả về thông tin hệ thống,
(3) triển khai các hàm hỗ trợ để đếm bộ nhớ trống và số tiến trình, và 
(4) viết chương trình test để kiểm tra.

\paragraph{Định nghĩa struct sysinfo}
Tạo file \texttt{kernel/sysinfo.h} chứa định nghĩa \texttt{struct sysinfo} với hai trường:
\texttt{uint64 freemem} (số byte bộ nhớ trống) và \texttt{uint64 nproc} (số tiến trình đang hoạt động).
Include file này trong \texttt{kernel/sysproc.c} để sử dụng.

\paragraph{Syscall sysinfo}
Triển khai \texttt{sys\_sysinfo(void)} trong \texttt{kernel/sysproc.c}: 
\begin{itemize}
    \item Lấy địa chỉ user space bằng \texttt{argaddr(0, \&addr)}
    \item Thu thập thông tin hệ thống: \texttt{info.freemem = getfreemem()} và \texttt{info.nproc = getnproc()}
    \item Copy struct ra user space bằng \texttt{copyout(p->pagetable, addr, (char *)\&info, sizeof(info))}
    \item Trả về -1 nếu copyout thất bại, ngược lại trả về 0
\end{itemize}

\paragraph{Hàm getfreemem}
Trong \texttt{kernel/kalloc.c}, triển khai \texttt{getfreemem()} để đếm số byte bộ nhớ trống:
\begin{itemize}
    \item Sử dụng lock \texttt{acquire(\&kmem.lock)} để đảm bảo thread-safe
    \item Duyệt qua danh sách liên kết \texttt{kmem.freelist}
    \item Với mỗi page trống, cộng thêm \texttt{PGSIZE} (4096 bytes) vào biến đếm
    \item Giải phóng lock \texttt{release(\&kmem.lock)} và trả về tổng số byte
\end{itemize}
Khai báo prototype \texttt{uint64 getfreemem(void);} trong \texttt{kernel/defs.h}.

\paragraph{Hàm getnproc}
Trong \texttt{kernel/proc.c}, triển khai \texttt{getnproc()} để đếm số tiến trình đang hoạt động:
\begin{itemize}
    \item Duyệt qua mảng \texttt{proc[NPROC]} chứa tất cả các tiến trình
    \item Với mỗi tiến trình, sử dụng lock \texttt{acquire(\&p->lock)} để truy cập an toàn
    \item Kiểm tra \texttt{if(p->state != UNUSED)} và tăng biến đếm nếu tiến trình đang được sử dụng
    \item Giải phóng lock \texttt{release(\&p->lock)} sau mỗi lần kiểm tra
    \item Trả về tổng số tiến trình
\end{itemize}
Khai báo prototype \texttt{uint64 getnproc(void);} trong \texttt{kernel/defs.h}.

\paragraph{Chương trình test sysinfotest}
Trong \texttt{user/sysinfotest.c}, triển khai chương trình kiểm tra:
\begin{itemize}
    \item Hàm \texttt{sinfo()} gọi system call \texttt{sysinfo()} và kiểm tra lỗi
    \item Hàm \texttt{testmem()} thực hiện các bước test:
    \begin{itemize}
        \item Gọi \texttt{sinfo(\&info)} để lấy thông tin hệ thống ban đầu
        \item Kiểm tra \texttt{info.nproc > 0} và \texttt{info.freemem > 0}
        \item Cấp phát thêm bộ nhớ bằng \texttt{sbrk(20000)}
        \item Gọi lại \texttt{sinfo(\&info1)} để lấy thông tin sau khi cấp phát
        \item Kiểm tra \texttt{info1.freemem < info.freemem} để đảm bảo bộ nhớ trống đã giảm
        \item In \texttt{"sysinfotest: OK"} nếu tất cả các kiểm tra đều pass
    \end{itemize}
\end{itemize}
Thêm \texttt{\$U/\_sysinfotest} vào \texttt{UPROGS} trong \texttt{Makefile} để biên dịch chương trình.
